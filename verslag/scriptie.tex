%Preambule met standaardinstellingen
\documentclass[a4paper,oneside]{book}
%Noot: zorg ervoor dat Nederlandse woordsplitsing geactiveerd is.
\usepackage[dutch]{babel}
% Noot: je kan het graphicxpakket een optie dvips of pdftex doorgeven
% in dat geval oet je ze ook aan iiiscriptie doorgeven, dus bijvoorbeeld
% \usepackage[dvips]{graphicx}
% \usepackage[dvips]{iiiscriptie}
\usepackage{graphicx}
\usepackage{iiiscriptie}
%Nuttig pakket voor URL's
\usepackage{url}
%
%Invullen velden voor titelpagina.
%
%eventueel bedrijfslogo voor een scriptie buiten de Hogeschool:
% Voorbeeld:
%\bedrijfslogo{%
%\includegraphics[totalheight=15mm]{logohogent}\\
%Naam Van Een Ander Bedrijf
%}
\departement{Departement Toegepaste Ingenieurswetenschappen}
\deptadres{Schoonmeersstraat 52 - 9000 Gent}
%\studiejaar{}
\soortrapport{
Masterproef voorgedragen tot het behalen van het diploma van\\
MASTER IN DE INDUSTRI\"ELE WETENSCHAPPEN: INFORMATICA\\
}
\title{Handleiding voor het gebruik van het III-scriptiepakket}
\author{Jan CNOPS}
\promotor{Iemand Anders}
%Noot: het argument van \promotoren wordt automatisch in een tabel gestoken.
\promotoren{
Intern:& De Ene\\
Extern:& De Andere\\
       & Nog Iemand
}
\begin{document}
\maketitle
\pagenumbering{roman}
\tableofcontents
\addcontentsline{toc}{chapter}{Inhoudsopgave}
\chapter*{Voorwoord}
%
% Standaard laat de bookstijl \chapter* uit de inhoudsopgave
%
\addcontentsline{toc}{chapter}{Voorwoord}
De huidige stijl wijkt op verschillende punten enigszins af van de richtlijnen uit het
Rode Boekje (o.a. in de titel van dit Voorwoord). De verschillen kunnen verantwoord
worden door vergelijking met de algemene praktijk. Zo hebben we ervoor gekozen om de
standaardinstellingen van Latex voor bibliografische verwijzingen te gebruiken, in
afwijking van het Rode Boekje.

Het is zeker niet de bedoeling dat deze tekst uitgroeit tot een handleiding LaTeX. Enkel
een paar punten specifiek voor de Hogeschool of voor informaticastudenten worden hier
vermeld. Interessante info over LaTeX zoals gebruikt door Gentse masterstudenten,
inclusief een handleiding:
\url{\http://latex.ugent.be}.
\chapter{Gebruik}
\pagenumbering{arabic}
\section{installatie}
Eenvoudige installatie: dump volgende bestanden in de directory waar je wil werken:
\begin{enumerate}
\item[-]\texttt{logohogent.eps}
\item[-]\texttt{logohogent.pdf}
\item[-]\texttt{iiiscriptie.sty}
\item[-]\texttt{scriptie.tex}
\end{enumerate}
Gebruik \texttt{scriptie.tex} als sjabloon om te beginnen.

Nette installatie: je kan ook de elementen van het pakket op de ge"eigende plaatsen van
je \TeX-installatie onderbrengen. Raadpleeg hiervoor de handleiding.
\section{Opmerkingen}
\begin{enumerate}
\item Het pakket gaat ervan uit dat de afmetingen van marges niet veranderd werden.
\item Het pakket veronderstelt dat het {\tt graphicx}pakket gebruikt wordt, met ofwel de
optie {\tt pdftex}, ofwel de optie {\tt dvips}, ofwel zonder optie. Dezelfde optie moet 
doorgegeven worden aan het {\tt iiiscriptie}pakket. 
De gemakkelijkste manier is om de standaardstijl van Latex te gebruiken. Dit geeft 
verwijzingen van de vorm \cite{Mmils}.
Het gebruik van \textsc{Bib}\TeX vergroot natuurlijk de mo(g)e(i)lijkheden.
\item Ongenummerde hoofdstukken, zoals de bibliografie en de inhoudstabel, worden door
LaTeX niet in de inhoudstabel opgenomen. Vandaar dat je het commando
\verb@\addcontentsline@ gebruikt.
\end{enumerate}
Succes ermee.
\appendix
\chapter{Nuttige pakketten}
Er zijn honderde, zoniet duizende pakketten in LaTeX om vanalles en nog wat te doen.
Enkele zijn in de preambule opgenomen:
\begin{enumerate}
\item {\tt babel}: zorgt voor taalspecifieke eigenschappen. Belangrijk is dat de meeste
standaardinstallaties niet weten hoe ze Nederlandse woorden moeten splitsen (kijk in je
log: als er staat {\tt ``hyphenation patterns for ... dutch... loaded''} is het OK,
anders niet). In teTex kan je de splitsingspatronen aanpassen met het {\tt texconfig}
lijncommando. In MikTeX: via het menu {\tt start -> programma's} kan je een aantal
instellingen veranderen.
\item {\tt graphicx}: voor afbeeldingen. Als je zelf tekeningen maakt: nooit naar
JPEG-bestanden converteren, dit levert onduidelijke afdrukken (=dezelfde kwaliteit als
MS-Word). Acceptabel is pdf of PostScript. Let op: wie omzet van dvi naar ps met dvips
kan geen pdf-bestanden gebruiken, en wie omzet naar pdf met pdflatex geen PostScript
(aleerhande omwegen leveren wel oplossingen).
\item  {\tt url}: URL's worden, zoals op \url{http://www.math.uiuc.edu/~hildebr/tex/bibliographies.html}
wordt uitgelegd, gezet met het \verb@\url{}@-com\-man\-do.
\end{enumerate}

%
% Bibliografie: titel veranderd in literatuurlijst
%
\def\bibname{Literatuurlijst}
%
% Het argument na \begin{thebibliography} moet (als tekst) even lang zijn als het
% langste label.
\begin{thebibliography}{Mills 1975}
%
% Standaard laat de bookstijl \chapter* uit de inhoudsopgave
%
\addcontentsline{toc}{chapter}{\bibname}
\bibitem{Mmils} {\sc Mills, Magnus,} {\it The scheme for full Employment},
Harper, 2004.
\bibitem{Boll} {\sc B\"oll, Heinrich,} {\it Und sagte kein einziges Wort}, ein Ulstein Buch,
1975.
\end{thebibliography}
\end{document}

